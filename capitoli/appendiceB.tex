%\section{Loop Quantum Gravity in (1,2)--signature}

LQG in three dimensions could be interesting, even just from a mathematical viewpoint. Moreover, it offers a remarkable laboratory to test ideas of loop quantisation---see Section 9.6 of \cite{pullin2} and Section 9.3.1 of \cite{rov1}.\\

The starting point will be a Holst--like theory on a 3--dimensional spacetime $\M^{1,2}$; the Lagrangian must be a $3$--form on $\J^1\Ci$ with $\Ci=\mathcal{F}(P)\times_{\M^{1,2}}\Con(P)$, for a principal $\Spin(1,2)$--bundle $P\xrightarrow{\pi}\M^{1,2}$ not depending on any Holst parameter! i.e., up to physical constants, it is

$$\mathbf{L}=R^{ij}\wedge e^k\,\epsilon_{ijk}\textcolor{gray}{-\frac{\Lambda}{6}e^i\wedge e^j\wedge e^k\,\epsilon_{ijk}}$$
which produces a variation
\begin{align*}
    \delta\mathbf{L}&=\epsilon_{ijk}\Biggl(\delta R^{ij}\wedge e^k+R^{ij}\wedge\delta e^k-\tiny{\frac{\Lambda}{6}}\bigg(\delta e^i\wedge e^j\wedge e^k+e^i\wedge\delta e^j\wedge e^k+e^i\wedge e^j\wedge\delta e^k\bigg)\Biggl)\\
    &=\epsilon_{ijk}\left(\nablaw\delta\omega^{ij}\wedge e^k+R^{ij}\wedge\delta e^k+\frac{\Lambda}{2}\, e^i\wedge e^j \wedge\delta e^k\right)\\
    &=\epsilon_{ijk}\left(\left(R^{ij}+\frac{\Lambda}{2}\,e^i\wedge e^j\right)\wedge\delta e^k-\nablaw e^k\wedge\delta\omega^{ij}\right)
\end{align*}
and $\delta\mathbf{S}=0$ implies by Theorem \ref{least_action} field equations
\begin{equation}
    \begin{cases}
    R^{ij}\textcolor{gray}{+\frac{\Lambda}{2}\,e^i\wedge e^j}=0\\
    \nablaw e^k=0
    \end{cases}
\end{equation}
Even if they characterise spacetimes in dimension $3$ to be flat, the theory is nevertheless non--trivial if the space--like surfaces have non--trivial global topology.\\

At this stage we could discuss either the (possibly infinite--dimensional) irreducible representations of $\Spin(1,2)\cong\SL(2,\R)$, in order to get \emph{spin foams} for a $3$--dimensional spacetime, or the finite--dimensional representations of $\Spin(2)\cong\U(1)$ to describe spin networks of a $2$--dimensional space.

The theory of representations of \(\mathfrak{sl}(2,\mathbb{R})\) is well established. Considering that \(\U(1) \subseteq \SL(2,\mathbb{R})\), any representation of \(\SL(2,\mathbb{R})\) must also be a representation of \(\U(1)\). The representations of \(\U(1)\), being compact, are classified in a manner similar to those of \(\SU(2)\).

In LQG literature, there is an ad--hoc prescription to construct spin foams in the case of \(\SL(2,\mathbb{C})\) along similar lines. Investigating the connections between these two approaches can be interesting.

%\subsection{\texorpdfstring{$\Spin(2)$}{a}--reduction of \texorpdfstring{$\Spin(1,2)$}{a}--bundles}

%$\Spin(2)\cong\U(1)\cong\SO(2)\cong\S^1$ and $\Spin(1,2)\cong\SL(2,\R)$. Qui possiamo implementare le rappresentazioni (infinito dimensionali) di $\SL(2,\R)$ per ottenre spinfoams in segnatura $(1,2)$, oppure possiamo implementare le rappresentazioni (finito dimensionali) di $\U(1)$ per descrivere spin--networks in segnatura $(1,2)$.   
%$$\vdots$$

%$$\L\left(\e,j^1\A,j^1\kappa\right)=\frac{1}{\boldsymbol{\kappa}}\bigg(2\nabla\kappa^i\wedge e^k\epsilon_{i0k}+\epsilon^{ij}_kF^k\wedge e^k\,\epsilon_{ijk}+2\kappa^i\wedge\kappa^j\wedge\kappa^k\,\epsilon_{ijk}\bigg)$$

%$$F^k_{ab}=\der_a\A_b^k-\der_b\A_a^k-\epsilon_{ij}^k\A_a^i\A_b^j$$

%$$p^{ab}_k=\frac{1}{\boldsymbol{\kappa}}\epsilon^{ij}_k\epsilon_{ijk}\,e^k\,\epsilon^{ab}$$

%$$\vdots$$

