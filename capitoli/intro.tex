The quest to reconcile GR and QM has been a longstanding challenge in mathematical and theoretical physics, representing a fundamental pursuit to unify our understanding of the universe at both macroscopic and microscopic scales. Historically, the conceptual and mathematical disparities between GR, which elegantly describes gravity as the curvature of spacetime, and QM, which deals with the probabilistic behavior of particles and fundamental forces, have led to numerous unsuccessful attempts to forge a coherent synthesis between the two theories.

In its first naive physical formulation, based on the equivalence principle and the general covariance principle, Einstein's GR perfectly fitted with the fiber bundle formalism of a variational field theory on a spacetime. Field theories based on variational principles and configuration bundles, particularly natural and gauge--natural theories, wield immense power in our understanding of fundamental physics. These theories provide a powerful framework for describing the behavior of fields and their interactions in a consistent and mathematically elegant manner. By treating fields as sections of configuration bundles, we can elegantly describe their dynamics and interactions through variational principles. This approach allows us to formulate field equations that govern the evolution of fields while respecting the underlying symmetries and constraints of the system. Natural and gauge--natural theories further enhance this framework by incorporating additional mathematical structures that capture the inherent symmetries and gauge transformations of the physical system. In natural theories, the action functional is constructed to be invariant under diffeomorphisms, ensuring that physical observables remain unchanged under smooth coordinate transformations. Gauge--natural theories extend this idea by introducing gauge transformations that encode the local symmetries of the system, such as gauge fields in electromagnetism or Yang--Mills theories. 

Yang--Mills theories serve as a prototype for generally covariant quantum field theories, with gauge connections as dynamical fields, the quantisation process involving their holonomies representation.  For many years after Einstein and Hilbert defined GR, the role of connections has been obscure in nature. Christoffel and Riemann already knew the later called Christoffel symbols (in 1869, when covariant differentiation was introduced). However, it was Levi Civita and Ricci Curbastro who, in 1900, essentially invented tensor calculus, laying the mathematical foundation necessary for the formulation of GR. They worked until 1917 to define parallel transport---as well as the Levi Civita connection that has Christoffel symbols as coefficients. One could argue that parallel transport is the first hint about the true nature of connections on manifols. Then Cartan (1926) generalised to more general connection forms, and, while Koszul defined connections on vector bundles, Ehresmann defined them on principal bundles in 1950. At that point the story of gauge fields has begun. Recently there has been a lot of work in formulating gauge--theories (as much as possible) completely in terms of holonomies around loops; this approach---called loop holonomy representation of a gauge--theory---has been used in quantum chromodynamics starting in the late 1970s. In 1990, Rovelli and Smolin \cite{smolin} published a paper in which they used Ashtekar's formulation of GR as an $
\SL(2,\C)$--gauge theory to construct a loop representation of quantum gravity.\quad This present work aims to be a spinoff of these papers, based on the latest mathematical developments, reviewing the mathematical foundations by following \cite{FFR1}, \cite{FFR2}, to attempt a rigorous mathematical background to the theory of LQG. %and maybe it also aims to give some original computations on spin foams. 

In a nutshell, LQG is a covariant background--independent non--perturbative quantum field theory for gravity and it aims to be a quantisation of GR. The fundamental challenges that the theory makes arise are many:  GR is ranked as a classical field theory for the gravitational interaction, i.e. a so--called gravitational theory, and what is available to quantize such theories is the quantum theory of fields. QFT essentially extends the way on which QM describes a single particle to a continuous distribution of particles throughout the whole spacetime: one starts with an empty spacetime, then one fills it with (classical, then) quantum fields satisfying some symmetries and, by eventually allowing such fields to interact with each other, QFT makes it possible to predict with astounding precision the phenomena that govern the physical world, as a synthesis of all possible scenarios at the microscopic scale---except for gravity.

The quantum interpretation offered by QFT is of a particles set in a (possibly curved) spacetime and takes into account gauge symmetries without a doubt, but says nothing about different symmetries such as those imposed by general covariance, which heavily characterizes GR.  The central objective of this work is to effectively blend the fundamental aspects of General Relativity—specifically, its general covariance and background independence—into a quantum framework. The aim of this thesis is to explore the mathematical formalism of LQG and analyze how it articulates in the attempt to provide a coherent quantum description of spacetime and gravity. However, it is important to emphasize that the focus of this work will specifically be on describing the geometric observables still on space.\\ 

The work is structured into three main chapters. The first chapter begins with a discussion on the general theory of connections on fiber bundles with only a differentiable structure, Ehresmann--like. It includes Levi--Civita and principal connections as particular cases. The chapter continues with an examination of classical field theory in its variational formulation, focusing on relativistic and gauge theories and addressing the principles of general and gauge covariance. Background--independence is then explored as an extension of general covariance, citing key results on Cauchy problems in a relativistic framework. The chapter concludes with an exploration of the Hamilton--Jacobi theory in a covariant context.

The second chapter is devoted to analyzing GR in its classical metric--formulation, using apropos the metric as dynamic variable. The field equations are derived from the Einstein--Hilbert Lagrangian and the well--posedness of the Cauchy problem is discussed. The theory of tetrads and spin frames is introduced next, with a rigorous derivation of the local form of spin connections and their curvature. Following this, Holst's gravitational theory is studied using tetrad and spin connection variables, demonstrating its dynamical equivalence with GR and establishing a gauge theory for the gravitational field in the $\SL(2,\C)$ group. This is further reduced to a gauge theory for the $\SU(2)$ group, known as the Ashtekar--Barbero--Immirzi (ABI) model for GR, using differential geometry arguments. Constraints are then derived and quantized using the covariant Hamilton--Jacobi theory, emphasizing that the quantisation model may only look at the boundary equations, rather than the bulks. Finally, the dynamical variables of the theory are restricted to the spatial slices of a globally hyperbolic spacetime.

In the third chapter LQG is presented as a generally covariant quantum field theory, resulting from the canonical quantization of the ABI field theory. The states of the theory are defined as discrete gauge--invariant $\SU(2)$ principal connections, and the finite--dimensional irreducible representations of the gauge group are studied in a general context. The theory of unitary representations of $\SU(2)$ allows for defining a Hilbert space structure on the space of invariant states, with elements of this space being spin networks that describe the quantum states of the spatial part of the gravitational field. The focus then shifts to $4$--valence spin networks, demonstrating how LQG modeled on these states is equivalent to a quantum theory of classical tetrahedra geometries. A $6$--dimensional Poisson algebra of invariant operators (observables) is constructed, representing the geometric properties of these \emph{quantum tetrahedra} and various examples of maximal commuting sub--algebras, which turn out $5$--dimensional, are analyzed. 

%The thesis concludes by addressing open questions. It considers the potential of CW complex theory as a more suitable tool for the study of quantum tetrahedra than the simplex theory currently used. Additionally, it explores the possibility of directly constructing the Hilbert space of spin foams using the theory of infinite-dimensional irreducible representations of SL(2,C), applying a procedure similar to the construction of spin networks. This novel approach could provide a better understanding of the gravitational field at the Planck scale, compared to the current a-posteriori path-integral approach, which is not yet fully understood in the field of Physics research. 
This thesis hopes to provide the reader with a comprehensive mathematical review of the state--of--the--art of LQG theory and glimpses an exploration of new potential directions for this frontier research.\\



%The work presented in this thesis is organized into three distinct chapters, each focusing on a specific aspect of the formulation and analysis of LQG. The first chapter introduces the fundamental mathematical concepts that underpin the theory, providing a solid foundation for understanding the subsequent theoretical developments. The second chapter examines the formulation of GR in a manner suitable for quantization as a generally covariant quantum field theory. Finally, the third chapter outlines in detail Loop Quantum Gravity up to the theory of knotted spin networks, as well as the quantum geometries of tetrahedra in spacetime.

%In conclusion, this thesis aims to offer a comprehensive and in-depth overview of Loop Quantum Gravity, with a focus on the geometric observables of spacetime as a focal point of inquiry. Through a detailed analysis of the mathematical concepts and physical implications of LQG, this research contributes to the ongoing discourse on the quest for a coherent quantum theory of gravity and opens new avenues for further theoretical and experimental developments in the field of fundamental and mathematical physics.

As a final note, be aware that the content of Chapter 1, up to Section 1.4, is included for self--contained purposes, providing essential background information on the theory of bundles, connections, and their curvatures, which are extensively employed in LQG and are formalized here, in the most possible general way I could. However, readers familiar with the material may choose to skip directly to the discussion on Holst theory in Chapter 2.